%% ------------------------------ Abstract ---------------------------------- %%
\begin{abstract}

Maize had no major artificial phosphorus input during its evolution in domestication and breeding until modern industrial agriculture.
Currently yield  relies on a steady but limited supply of mined phosphate fertilizer.
Even so,  traditional farmers in Mexico still successfully grow varieties with no addition of industrial fertilizer. Understanding the nature of the genetic variation underlying this adaptation can contribute to the production of phosphorus efficient crops that could lower costs and water pollution from leached fertilizers.
In this study I establish genome environment associations (GEA) between soil phosphorus availability, from interpolated soil profile maps, and the genetic diversity of a collection of ~2000 open-pollinated varieties in Mexico. Here I find evidence of association between a chromosomal inversion, inv4m, and multiple soil phosphorus parameters, specially, water soluble phosphorus and phosphorus retention.
Although this inversion is presumed to be adaptive to high elevation in Mexico no current measurements of fitness exists where inv4m has been experimentally isolated from background genetic variation. To test whether inv4m is adaptive, I used a segregating HIF population for inv4m, derived from a cross of the temperate inbred B73 and  the Mexican Andosol variety MICH21, grown in the field under high and low phosphorus treatments.
In addition to this I built both an expression and a phospholipid profile from a developmental series of leaves. Complementing the GEA approach I made a QTL search for seed phosphorus content using ionomics. or this I used 8 families of BC2S3 RILs, derived from parents traditionally cultivated in Mexican Andosols and the tropical line CML530.
Results show that inv4m accelerates flowering independent of phosphorus treatments at our test field in Pennsylvania. This the first direct experimental evidence of the adaptive contribution of  inv4m, however the adaptation is rather global  in character, with respect to soil phosphorus availability.
As inv4m does not show the pattern of Gene X Environment interaction expected in the case of local adaptation. Currently I am doing the gene expression analysis o test the efect of inv4m on gene expression pre-reproductive leaves. In our multiparental population I was able to detect 2 QTLs for seed phosphorus content in chromosomes 7 and 8, furthter analyses are required to characterize possible candidates. 
\end{abstract}


%% ---------------------------- Copyright page ------------------------------ %%
%% Comment the next line if you don't want the copyright page included.
\makecopyrightpage

%% -------------------------------- Title page ------------------------------ %%
\maketitlepage

%% -------------------------------- Dedication ------------------------------ %%
%\begin{dedication}
% \centering To my parents.
%\end{dedication}

%% -------------------------------- Biography ------------------------------- %%
% \begin{biography}
% The author was born in a small town \ldots
% \end{biography}

%% ----------------------------- Acknowledgements --------------------------- %%
%\begin{acknowledgements}
%Thanks to family, advisor, committee, office friends, etc. ...
%\end{acknowledgements}


\thesistableofcontents

% \thesislistoftables

\thesislistoffigures
